\chapter*{Einleitung}

 Bei der alltäglichen Kommunikation im Internet in Form von 
 Überweisungen, Onlinebestellungen oder Versenden von Nachrichten 
 existieren oft Institutionen oder Organisationen, die sich
 zwischen Sender und Empfänger befinden. Eine Bank ist beispielsweise eine solche Institution, die bei einer Kommunikation zwischen A und B zwischengeschaltet ist und die Überweisung des Geldes organisiert und durchführt. A muss sich somit unmittelbar an die Bank wenden, um das Geld an B zu überweisen. Der Nachteil dieser Konzeption ist, dass A ein gewisses Vertrauen in die Bank haben muss und diese gegebenenfalls Gebühren für das Transferieren des Geldbetrags verlangt. 

 Wäre es also nicht sinnvoll, ein System zu verwenden, welches Dritte bei einer Kommunikation zwischen zwei Parteien irrelevant macht und Person A und B direkt miteinander in Kontakt treten lässt? Ein solches System wurde mit der Erschaffung von Bitcoin bereitgestellt. Bitcoin ermöglicht es in Form eines Peer-to-Peer-Systems, Online-Zahlungen von zwei an der Transaktion teilnehmenden Parteien durchzuführen, ohne ein außenstehendes Finanzinstitut miteinzubeziehen \footnote{\parencite{Satoshi.}}. Ein solches System wird mithilfe einer sogenannten Blockchain realisiert. Eine Blockchain ist eine Art dezentrale Datenbank, die mehrere Einträge zu einem Block zusammenfasst und fortlaufend an die bestehende Kette (Chain) anhängt \footnote{\parencite[vgl.][S. 17]{UK.}}.


Wie genau diese Kette technisch umgesetzt wird, woraus ein solcher Block besteht und welche Voraussetzungen erfüllt sein müssen, damit dieser an die Kette angehängt wird, wird in den nächsten Seiten erläutert.
