\chapter{Einleitung}

 Bei der alltäglichen Kommunikation im Internet in Form von 
 Überweisungen, Onlinebestellungen oder Versenden von Textnachrichten 
 existieren oft Organisationen, die sich
 zwischen Sender und Empfänger befinden. Eine Bank ist beispielsweise eine solche Institution, die bei einer Kommunikation zwischen A und B zwischengeschaltet ist und die Überweisung des Geldes organisiert und durchführt. A muss sich somit unmittelbar an die Bank wenden, um das Geld an B zu überweisen. Der Nachteil einer solchen Kontrollinstanz ist, dass eine Überweisung meist ein bis zwei Tage dauert und ein gewisses Vertrauen erbracht werden sollte.

 Deshalb wäre ein System sinnvoll, welches Dritte überflüssig macht und die betroffenen Personen direkt miteinander in Kontakt treten lässt. Ein solches System wurde mit der Erschaffung von Bitcoin bereitgestellt. Bitcoin ermöglicht es in Form eines Peer-to-Peer-Systems, Online-Zahlungen von zwei an der Transaktion teilnehmenden Parteien durchzuführen, ohne ein außenstehendes Finanzinstitut miteinzubeziehen \footnote{\parencite{Satoshi.}}. Ein solches System wird mithilfe einer sogenannten Blockchain realisiert. Eine Blockchain ist eine Art dezentrale Datenbank, die mehrere Daten zu einem Block zusammenfasst und fortlaufend an eine bestehende Kette (Chain) anhängt, die aus weiteren Blöcken besteht \footnote{\parencite[vgl.][S. 17]{UK.}}.


Wie genau diese Kette technisch umgesetzt wird, woraus ein solcher Block besteht und welche Voraussetzungen erfüllt sein müssen, damit dieser an die Kette angehängt wird, wird in den nächsten Seiten erläutert.
